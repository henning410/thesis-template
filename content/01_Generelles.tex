\section{Generelles zu Abschlussarbeiten}
\subsection{Unterschiede zwischen Deutsch- und Englischer Abschlussarbeit}
Generell unterscheidet sich das Schreiben auf Englisch nicht groß vom Schreiben auf Deutsch. Allerdings gibt es ein paar Feinheiten, deren du dir bewusst sein solltest, wenn du mit dem Schreiben einer Arbeit auf Englisch beginnst.
\begin{table}[H]
    \centering
    \begin{tblr}{
            width = \linewidth,
            hlines,
            vlines,
        }
        \textbf{Deutsch}              & \textbf{Englisch}     \\
        Nominalstil                   & Reich an Verben       \\
        Sprache oft im Passiv         & Sprache im Aktiv      \\
        Oft langer, komplexer Satzbau & Kurze, einfache Sätze \\
    \end{tblr}
    \caption{Unterschiede zwischen Deutsch und Englischer Abschlussarbeit}
    \label{Unterschiede zwischen Deutsch und Englischer Abschlussarbeit}
\end{table}\noindent
Du solltest dir beim Schreiben deiner Arbeit darüber im Klaren sein, ob du in britischem oder auf amerikanischem Englisch schreibst.
Die Unterschiede sind groß und Einheitlichkeit ist sehr wichtig für das Gesamtbild deiner Abschlussarbeit.
Dass Nomen im Englischen nicht großgeschrieben werden, solltest du wissen.
Aber es gibt noch mehr Unterschiede im akademischen Englisch, so werden zum Beispiel, je nachdem welches Format du nutzt, in Überschriften oft fast alle Wörter ‚capitalized‘.



\subsection{Aufbau einer Abschlussarbeit}
Generell solltest du dir zuerst die Ziele deiner Arbeit überlegen. Daraus entsteht dann die Zielsetzung.
Außerdem solltest du eine Literaturrecherche machen, um deine Arbeit in den Kontext anderer Arbeiten zu setzen und die Problemstellung klar herausarbeiten zu können. Dein Aufbau deiner Thesis kann verschieden sein, jedoch habe ich folgenden Aufbau benutzt:
\begin{itemize}
    \item Introduction / Einleitung
          \begin{itemize}
              \item Problem Statement / Problemstellung
              \item Objectives / Ziele dieser Arbeit
              \item Structure of this Thesis / Aufbau dieser Arbeit
              \item Related Work / Verwandte Arbeiten
          \end{itemize}
    \item Fundamentals / Grundlagen
          \begin{itemize}
              \item Hier erklärst du grundlegende Technologien oder Begriffe, die für das Verständnis deiner Arbeit relevant sind
          \end{itemize}
    \item ... Der weitere Aufbau hängt stark davon ab, ob du etwas praktisches machst etc. Oft gliedert man in Design und Implementierung.
    \item Summary and Outlook / Zusammenfassung und Ausblick
          \begin{itemize}
              \item Results / Ergebnisse
              \item Discussion / Diskussion
              \item Conclusion / Fazit
              \item Future Work / Ausblick
          \end{itemize}
\end{itemize}