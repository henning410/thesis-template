\appendix
\section{Anhang}
\label{Code Appendix}

\subsection{Script to Search for GitHub Repositories}
\label{Script to search for GitHub repositories}
\autoref{Python script so search for GitHub repositories} is used to search on GitHub for repositories matching the keywords provided by the user.
In comparison to a manual search, this script offers several advantages.
Firstly, the output includes all repositories found, and secondly, the output produces a link to the repository, which can then be used for further processing.
We developed this script by making use of the capabilities of ChatGPT 3.5, a sophisticated large language model \cite{openai_chatgpt_2024}.
\lstinputlisting[language=Python, caption=Python script so search for GitHub repositories, label=Python script so search for GitHub repositories]{git_search.py}


\newpage
\subsection{Script to Process Output Produced by RestTestGen}
\label{Script to process output produced by RestTestGen}
Since RestTestGen generates a \ac{JSON} file for every request sent, we need to search through these files to identify any errors found by RestTestGen.
To automate this process, we developed a Python script shown in \autoref{Python script to search inside RestTestGen output folder for found errors}.
The script takes two parameters: the directory containing the \ac{JSON} files and the search term to look for within these files.
To search for errors in the files, we use the search term "FAIL".
We developed this script by making use of the capabilities of ChatGPT 3.5, a sophisticated large language model \cite{openai_chatgpt_2024}.
\lstinputlisting[language=Python, caption=Python script to search inside RestTestGen output folder for found errors, label=Python script to search inside RestTestGen output folder for found errors]{string_search.py}